
\begin{abstract}


% Unlike news, the scientific text usually requires domain expertise to understand. In recent years, NLP methods have leveraged domain expertise to improve the quality of IE on biomedical texts with huge impact.

% Unlike news, scientific text often requires multimodal domain expertise to understand, such as molecule knowledge for chemistry text. While Natural Language Processing methods have leveraged domain knowledge to improve the quality of Information Extraction on biomedical texts with huge impact, little work has been done in core chemistry literature, which forms the foundation of many biomedical research. In this work, we give a thrust to this underexplored yet important domain by introducing CHEMET, a benchmark dataset in fine-grained entity typing, and an effective method that incorporates external multimodal knowledge. Throughout our experiments, we showed that our method outperforms the state-of-the-art models in entity typing. To the best of our knowledge, this is the first work on tackling the problem of fine-grained chemical entity typing.

How to extract knowledge about chemical reactions from the core chemistry literature is a new emerging challenge that has not been well studied. In this paper, we introduce a new benchmark data set (CHEMET) to facilitate the study of knowledge extraction in this new domain. Fine-grained chemical entity typing poses interesting new challenges especially because of the complex name mentions frequently occurring in chemistry literature and graphic representation of entities. At the same time, there are also interesting new opportunities to leverage external chemistry knowledge resources. We propose a novel multi-modal representation learning framework to solve the problem of fine-grained chemical entity typing by leveraging external resources with chemical structures and using cross-modal attention to learn effective representation of text in the chemistry domain. Experiment results show that the proposed framework outperforms multiple state of the art. 



% ... or show that the proposed framework can effectively exploit external knowledge to improve accuracy of entity typing...

% Information extraction in scientific text is vastly different from news due to the amount of need for domain knowledge. While there was a surge of information extraction works on biomedical text in the past few years, 

% Information e xtraction refers to the task of transforming text into structured knowledge elements, which is a crucial step for many downstream tasks, such as knowledge graph construction and question answering. As the amount of research literature is growing exponentially, 
% Chemistry information e

% Compared to the news domain, information
% extraction (IE) from biomedical text requires
% much broader domain knowledge. However,
% many previous IE methods do not utilize
% any external knowledge during inference. Due


% This document is a supplement to the general instructions for *ACL authors. It contains instructions for using the \LaTeX{} style files for EMNLP 2021. 
% The document itself conforms to its own specifications, and is therefore an example of what your manuscript should look like.
% These instructions should be used both for papers submitted for review and for final versions of accepted papers.
\end{abstract}


